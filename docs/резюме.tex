\documentclass{article}
\usepackage{cmap}
\usepackage[utf8]{inputenc}
\usepackage[russian]{babel}
\usepackage{setspace,amsmath}
\usepackage{lipsum}
\usepackage[usestackEOL]{stackengine}
\usepackage{lipsum}
\usepackage{kantlipsum}
\usepackage[unicode, pdftex]{hyperref}
\usepackage{ragged2e}
\usepackage{multicol}
\usepackage{setspace}
\usepackage{vwcol}
\usepackage[pdftex]{graphicx}
\usepackage[left=1cm,right=1cm, top=1cm,bottom=1cm,bindingoffset=0cm]{geometry}
\usepackage[usenames]{color}
\renewcommand{\baselinestretch}{1.1}
\graphicspath{{pictures/}}
\DeclareGraphicsExtensions{.pdf,.png,.jpg}
\newcommand\zz[1]{\par{\normalsize\strut #1} \hfill\ignorespaces}
\pagestyle{empty}
\begin{document}
\begin{vwcol}[widths={0.8,0.2},
 sep=.8cm, justify=flush,rule=0pt,indent=1em] 
\begin{spacing}{1.1}
\noindent\textbf{\Huge{Peganov Nikita}}\\
\end{spacing}
\noindent\textcolor[rgb]{0.1255,0.2902,0.7843}{\textbf{\Large{Backend Developer}}}\\
Backend developer with working experience in highload systems and project organization. Specialize in Java, Python and C++ with 3 years of production development experience. I am fond of creating IT-startups, generating ideas and learning new technologies. I prefer to work in Moscow.\\
\\
\noindent\textcolor[rgb]{0.1255,0.2902,0.7843}{\textbf{\Large{EXPERIENCE}}}\\
\begin{Large}
\textbf{Yandex}, Moscow
\end{Large}
\hspace{200pt}July 2022 — October 2022\\
\textbf{Intern Python/Java Developer} of a compensation calculation system in HR-Tech\\
• added processing of uploads to the system reducing user work time by 50\%\\
• implemented uploading of all user data to the cloud to increase system fault tolerance\\
• solved 23 minor tasks and wrote unit tests for them to improve the system\\
\\
\begin{Large}
\textbf{Yandex}, Moscow
\end{Large}
\hspace{190pt}July 2021 — November 2021\\
\textbf{Intern Python Developer} in traffic management team\\
• reduced by 50\% the load on the servers with DB by translating readonly requests to replicas\\
• added ipv6 processing into balancers’ work allowed to use new ip’s without extra 900 lines\\
• created an internal library redirecting the request to the nearest datacenter\\
\\
\textbf{\Large{Freelance}}
\hspace{260pt}July 2020 — July 2021\\ 
• \href{https://github.com/NikPeg/OzonParsing}{wrote scripts parsing sites for market analytics}\\
• \href{https://github.com/NikPeg/ExtraterrestrialBot}{launched vk bots for online games with a maximum load of 63 people}\\
• \href{https://github.com/NikPeg/dobrobot365}{developed a telegram bot for the 365 good deeds project}\\
\\
\noindent\textcolor[rgb]{0.1255,0.2902,0.7843}{\textbf{\Large{PROJECTS}}}\\
\begin{Large}
\textbf{Huawei}, Moscow
\end{Large}
\hspace{190pt}November 2021 — May 2022\\
\textbf{Python Developer}, Data Scientist\\
\href{https://github.com/NikPeg/Reinforcement-learning-for-resource-allocation-tasks-in-the-cloud}{Research work «Reinforcement learning for resource allocation tasks in the cloud»}\\
• tested the applicability of reinforcement learning in real conditions\\
• created and trained a neural network processing 4500 requests per hour\\
• the total number of processed requests was 1.5 million\\
\\
\begin{Large}
\href{https://www.canva.com/design/DAEaT9nPC7Y/y7_r2BzUEiUiFKW36oP-Pw/view?utm_content=DAEaT9nPC7Y&utm_campaign=designshare&utm_medium=link&utm_source=publishsharelink}{\textbf{Startup SmartLearning}, Moscow}
\end{Large}
\hspace{70pt}November 2020 — March 2021\\
\textbf{Kotlin Developer}, SEO \href{https://gitlab.com/peganov.nik/smartlearning}{the development of the Android-application for quick learning}\\
• assembled and managed a team of 4 programmers\\
• developed an Android application downloaded by 183 users\\
• \href{https://fest.hse.ru/top1002021}{took the 19th place at the HSE Fest startup competition}\\
\\
\begin{Large}
\textbf{Wikirace}, Summer Informatics School
\end{Large}
\hspace{85pt}June 2018 — July 2018\\
\textbf{Python Developer} in an \href{https://github.com/igoose1/wikirace}{online-game where you need to find a way between wikipedia pages}\\
• wrote a script that processed 1.8 million wikipedia articles\\
• implemented the back button which increased user convenience\\
• added multiplayer increasing the number of players by 30\%\\
\\
\noindent\textcolor[rgb]{0.1255,0.2902,0.7843}{\textbf{\Large{EDUCATION}}}\\
\begin{Large}
\textbf{\href{https://hse.ru/}{Higher School of Economics}}, Moscow
\end{Large}
\hspace{75pt}Since September 2020\\
\href{https://cs.hse.ru/}{Bachelor in Computer Science}\\
• Graduated courses with excellence: C++ Programming, Introduction to Applied Analytics, Databases, Calculus, Discrete Mathematics, English for Business Purposes\\
• Completed courses: Operating Systems, Software Design on Java, Algorithms and Data Structures, Computer System Architecture, Algebra, Software Development Methodologies, Probability Theory\\
•\href{https://github.com/NikPeg/synchronization-of-neuromorphic-networks-of-the-close-world-from-the-point-of-view-of-complexes} {Research works in the field of neural networks} were rated 9 out of 10 by scientific supervisors\\
\newpage
~\\
\noindent\textbf{Moscow, Russia}\\
\noindent\textbf{\textcolor[rgb]{0.1255,0.2902,0.7843}{\href{https://t.me/peganov\_ns}{t.me/peganov\_ns}}}\\
peganov.nik@gmail.com\\
+7 (977) 744-19-23\\
\href{https://vk.com/nikpeg}{vk.com/nikpeg}\\
\href{https://github.com/NikPeg}{github.com/NikPeg}\\
\href{https://gitlab.com/peganov.nik}{gitlab.com/peganov.nik}\\
\\
\noindent\textcolor[rgb]{0.1255,0.2902,0.7843}{\textbf{COMPUTER SKILLS}}\\
\textbf{Python} (\href{https://github.com/igoose1/wikirace}{Django}, \href{https://github.com/NikPeg/synchronization-of-neuromorphic-networks-of-the-close-world-from-the-point-of-view-of-complexes}{Jupyter},\\
\href{https://github.com/NikPeg/Reinforcement-learning-for-resource-allocation-tasks-in-the-cloud}{numpy}, \href{https://github.com/NikPeg/Calculus_and_Algebra_sympy}{sympy}, sklearn)\\
\textbf{Java} (\href{https://github.com/NikPeg/jigsaw_sockets_and_saving}{Sockets, JavaFX,}\\
\href{https://github.com/NikPeg/StudentsBooks}{Multithreading}, Derby)\\
\textbf{C/С++}\\
(\href{https://github.com/NikPeg/OS_multithreaded_tasks}{Multithreading}, STL)\\
\textbf{C\#}\\
(\href{https://gitlab.com/peganov.nik/messengerapi}{Requests}, \href{https://gitlab.com/peganov.nik/diveintofractal}{Windows forms})\\
\textbf{Kotlin} \href{https://gitlab.com/peganov.nik/smartlearning}{(Android)}\\
\textbf{\href{https://github.com/NikPeg/CoffeePult}{SQL}}\\
(\href{https://github.com/NikPeg/ExtraterrestrialBot}{SQLite}, \href{https://github.com/NikPeg/CoffeePult}{PostgreSQL})\\
\\
\noindent\textcolor[rgb]{0.1255,0.2902,0.7843}{\textbf{TECHNOLOGIES}}\\
\textbf{
Git\\
Linux\\
Docker\\
\href{https://github.com/NikPeg/csbot}{Selenium}\\
JetBrains IDEs\\
}
\\
\noindent\textcolor[rgb]{0.1255,0.2902,0.7843}{\textbf{LANGUAGES}}\\
\textbf{English} (B2)\\
\textbf{Russian} (Native speaker)\\
\\
\noindent\textcolor[rgb]{0.1255,0.2902,0.7843}{\textbf{ACHIEVEMENTS}}\\
\textbf{Winner} of \href{https://olymp.itmo.ru/}{ITMO "Open\\ Olympiad of schoolchildren in computer science"} (2020)\\
\textbf{Winner} of \href{https://olympiads.mccme.ru/ommo/23/}{"United\\Mathematical Olympiad of schoolchildren"} (2020)\\
\textbf{Medalist} of \href{https://olymp.mipt.ru/}{Olympiad of schoolchildren "Fiztech" in mathematics} (2020)\\
\textbf{Medalist} of regional stage in \href{https://vos.olimpiada.ru/}{All-Russian Olympiad of schoolchildren of\\
 CS and ICT} (2019)\\
\\
\noindent\textcolor[rgb]{0.1255,0.2902,0.7843}{\textbf{OTHER COURSES}}\\
Summer School of \href{https://raai.space/}{\textbf{Russian\\
Association of Artificial Intelligence}} (2021)\\
\href{https://fintech.tinkoff.ru/school/generation/}{\textbf{Tinkoff} Generation} (2019)\\
\href{https://myitschool.ru/}{\textbf{Samsung} IT-school} (2018)\\
\href{https://lksh.ru/}{\textbf{SIS}} (2016 — 2018)\\
\end{vwcol} 
\end{document}