\documentclass{article}
\usepackage{cmap}
\usepackage[utf8]{inputenc}
\usepackage[russian]{babel}
\usepackage{setspace,amsmath}
\usepackage{lipsum}
\usepackage[usestackEOL]{stackengine}
\usepackage{lipsum}
\usepackage{kantlipsum}
\usepackage[unicode, pdftex]{hyperref}
\usepackage{ragged2e}
\usepackage{multicol}
\usepackage{setspace}
\usepackage{vwcol}
\usepackage[pdftex]{graphicx}
\usepackage[left=1cm,right=1cm, top=1cm,bottom=1cm,bindingoffset=0cm]{geometry}
\usepackage[usenames]{color}
\renewcommand{\baselinestretch}{1.1}
\graphicspath{{pictures/}}
\DeclareGraphicsExtensions{.pdf,.png,.jpg}
\newcommand\zz[1]{\par{\normalsize\strut #1} \hfill\ignorespaces}
\pagestyle{empty}
\begin{document}
\begin{vwcol}[widths={0.8,0.2},
 sep=.8cm, justify=flush,rule=0pt,indent=1em] 
\begin{spacing}{1.1}
\noindent\textbf{\Huge{Пеганов Никита}}\\
\end{spacing}
\noindent\textcolor[rgb]{0.1255,0.2902,0.7843}{\textbf{\Large{Бэкенд-разработчик}}}\\
Разработчик бэкенда с опытом работы в высоконагруженных системах и организации проектов. Специализируюсь в языках Java, Python и C++, имею 3 года опыта производственной разработки. Я увлекаюсь созданием IT-стартапов, генерацией идей и изучением новых технологий. Предпочитаю работать в Москве или удалённо.\\
\\
\noindent\textcolor[rgb]{0.1255,0.2902,0.7843}{\textbf{\Large{ОПЫТ РАБОТЫ}}}\\
\begin{Large}
\textbf{Яндекс}, Москва
\end{Large}
\hspace{185pt}Июль 2022 — Октябрь 2022\\
\textbf{Стажер Python/Java разработчик} системы расчета компенсаций в HR-Tech\\
• добавил автоматическую систему выгрузок, сократив время работы пользователей на 50\%\\
• реализовал сохранение всех данных пользователей в облаке, повысив отказоустойчивость\\
• решил 23 задачи и написал юнит-тесты к ним для улучшения системы\\
\\
\begin{Large}
\textbf{Яндекс}, Москва
\end{Large}
\hspace{190pt}Июль 2021 — Ноябрь 2021\\
\textbf{Стажер Python разработчик} в команде управления трафиком\\
• сократил на 50\% нагрузку на сервер путем перевода readonly-запросов на его реплики\\
• добавил IPv6 в работу балансиров, чтобы использовать новые IP без лишних 900 строк\\
• создал внутреннюю библиотеку, перенаправляющую запросы в ближайший датацентр\\
\\
\textbf{\Large{Фриланс}}
\hspace{250pt}Июль 2020 — Июль 2021\\ 
• \href{https://github.com/NikPeg/OzonParsing}{писал скрипты для парсинга сайтов с целью анализа рынка}\\
• \href{https://github.com/NikPeg/ExtraterrestrialBot}{запускал ботов вконтакте для онлайн-игр с максимальной нагрузкой в 63 пользователя}\\
• \href{https://github.com/NikPeg/dobrobot365}{разработал telegram-бота для проекта 365 добрых дел}\\
\\
\noindent\textcolor[rgb]{0.1255,0.2902,0.7843}{\textbf{\Large{ПРОЕКТЫ}}}\\
\begin{Large}
\textbf{Huawei}, Москва
\end{Large}
\hspace{198pt}Ноябрь 2021 — Май 2022\\
\textbf{Python Разработчик}, Специалист по обработке данных\\
\href{https://github.com/NikPeg/Reinforcement-learning-for-resource-allocation-tasks-in-the-cloud}{НИР «Обучение с подкреплением для задач распределения ресурсов в облаке»}\\
• проверил применимость обучения с подкреплением в реальных условиях\\
• создал и обучил нейронную сеть, обрабатывающую 4500 запросов в час\\
• итоговое количество обработанных запросов составило 4.5 миллиона\\
\\
\begin{Large}
\href{https://www.canva.com/design/DAEaT9nPC7Y/y7_r2BzUEiUiFKW36oP-Pw/view?utm_content=DAEaT9nPC7Y&utm_campaign=designshare&utm_medium=link&utm_source=publishsharelink}{\textbf{Стартап SmartLearning}, Москва}
\end{Large}
\hspace{75pt}Ноябрь 2020 — Март 2021\\
\textbf{Kotlin Разработчик}, SEO \href{https://gitlab.com/peganov.nik/smartlearning}{разработка Android-приложения для быстрого обучения}\\
• собрал и руководил командой из 4 программистов\\
• разработал приложение для Android, загруженное 183 пользователями\\
• \href{https://fest.hse.ru/top1002021}{занял 19-е место на конкурсе стартапов HSE Fest}\\
\\
\begin{Large}
\textbf{Wikirace}, Летняя Компьютерная Школа
\end{Large}
\hspace{45pt}Июнь 2018 — Июль 2018\\
\textbf{Python Разработчик} в \href{https://github.com/igoose1/wikirace}{онлайн-игре по поиску путей между страницами Википедии}\\
• написал скрипт, который обработал 1,8 миллиона статей в Википедии\\
• реализовал кнопку возврата, повысившую удобство пользователей\\
• добавил мультиплеер, увеличивший количество игроков на 30\%\\
\\
\noindent\textcolor[rgb]{0.1255,0.2902,0.7843}{\textbf{\Large{ОБРАЗОВАНИЕ}}}\\
\begin{Large}
\textbf{\href{https://hse.ru/}{Высшая Школа Экономики}}, Москва
\end{Large}
\hspace{90pt}С сентября 2020\\
\href{https://cs.hse.ru/}{Бакалавр на Факультете Компьютерных Наук}\\
• Оконченные с отличием курсы: Программирование на C++, Введение в прикладную\\ аналитику, Базы данных, Дискретная математика, Английский язык для деловых целей\\
• Пройденные курсы: Операционные системы, Разработка ПО на Java, Алгоритмы и\\ структуры данных, Архитектура компьютерных систем, Методологии разработки ПО\\
•\href{https://github.com/NikPeg/synchronization-of-neuromorphic-networks-of-the-close-world-from-the-point-of-view-of-complexes} {Научные работы по изучению нейронных сетей} оценены руководителями на 9 из 10\\
\newpage
~\\
\noindent\textbf{Москва, Россия}\\
\noindent\textbf{\textcolor[rgb]{0.1255,0.2902,0.7843}{\href{https://t.me/peganov\_ns}{t.me/peganov\_ns}}}\\
peganov.nik@gmail.com\\
+7 (977) 744-19-23\\
\href{https://github.com/NikPeg}{github.com/NikPeg}\\
\href{https://gitlab.com/peganov.nik}{gitlab.com/peganov.nik}\\
\\
\noindent\textcolor[rgb]{0.1255,0.2902,0.7843}{\textbf{НАВЫКИ}}\\
\textbf{Python} (\href{https://github.com/igoose1/wikirace}{Django}, \href{https://github.com/NikPeg/synchronization-of-neuromorphic-networks-of-the-close-world-from-the-point-of-view-of-complexes}{Jupyter},\\
\href{https://github.com/NikPeg/Reinforcement-learning-for-resource-allocation-tasks-in-the-cloud}{numpy}, \href{https://github.com/NikPeg/Calculus_and_Algebra_sympy}{sympy}, sklearn)\\
\textbf{Java} (\href{https://github.com/NikPeg/jigsaw_sockets_and_saving}{Sockets, JavaFX,}\\
\href{https://github.com/NikPeg/StudentsBooks}{Multithreading}, Derby)\\
\textbf{C/С++}\\
(\href{https://github.com/NikPeg/OS_multithreaded_tasks}{Multithreading}, STL)\\
\textbf{C\#}\\
(\href{https://gitlab.com/peganov.nik/messengerapi}{Requests}, \href{https://gitlab.com/peganov.nik/diveintofractal}{Windows forms})\\
\textbf{Kotlin} \href{https://gitlab.com/peganov.nik/smartlearning}{(Android)}\\
\textbf{\href{https://github.com/NikPeg/CoffeePult}{SQL}}\\
(\href{https://github.com/NikPeg/ExtraterrestrialBot}{SQLite}, \href{https://github.com/NikPeg/CoffeePult}{PostgreSQL})\\
\\
\noindent\textcolor[rgb]{0.1255,0.2902,0.7843}{\textbf{ТЕХНОЛОГИИ}}\\
\textbf{
Git\\
Linux\\
Docker\\
\href{https://github.com/NikPeg/csbot}{Selenium}\\
JetBrains IDEs\\
}
\\
\noindent\textcolor[rgb]{0.1255,0.2902,0.7843}{\textbf{ЯЗЫКИ}}\\
\textbf{English} (B2)\\
\textbf{Russian} (носитель)\\
\\
\noindent\textcolor[rgb]{0.1255,0.2902,0.7843}{\textbf{ДОСТИЖЕНИЯ}}\\
\textbf{Победитель} \href{https://olymp.itmo.ru/}{"Открытой олимпиады ИТМО по\\программированию"} (2020)\\
\textbf{Победитель} \href{https://olympiads.mccme.ru/ommo/23/}{"Межвузовской Математической Олимпиады"} (2020)\\
\textbf{Призер} \href{https://olymp.mipt.ru/}{олимпиады\\школьников "Физтех"\\по математике} (2020)\\
\textbf{Призер} регионального этапа \href{https://vos.olimpiada.ru/}{Всероссийской\\олимпиады школьников по информатике} (2019)\\
\\
\noindent\textcolor[rgb]{0.1255,0.2902,0.7843}{\textbf{ДРУГИЕ КУРСЫ}}\\
Летняя школа\\\href{https://raai.space/}{\textbf{Российской Ассоциации Искусственного\\Интеллекта}} (2021)\\
\href{https://fintech.tinkoff.ru/school/generation/}{\textbf{Tinkoff} Generation} (2019)\\
\href{https://myitschool.ru/}{\textbf{Samsung} IT-школа} (2018)\\
\href{https://lksh.ru/}{\textbf{Летняя Компьютерная Школа}} (2016 — 2018)\\
\end{vwcol} 
\end{document}